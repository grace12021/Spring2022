\chapter{Cayley-Hamilton Thm and Nakayama's Lemma}
\section*{Reminder: Modules}
The reference of this reminder section is [1] and [2].
\subsection*{Modules and module homomorphisms}
Let us consider a commutative ring $R$ with $1$. An $R$-module is an Abelian group $M$ with a multiplication map
$$
R \times M \rightarrow M, \quad \hbox{written }(f,m) \mapsto f\cdot m
$$
satisfying following properties:\\
(i) $f \cdot (m\pm n) = f \cdot m \pm f \cdot n$, (ii) $(f \pm g) \cdot m = f \cdot m \pm g \cdot m$, (iii) $(fg) \cdot m = f \cdot (g \cdot m)$, (iv) $1_R \cdot m = m$\\
for all $f, g \in R$ and $m,n \in M$.\\
A subset $N \subset M$ is a submodule if $f \cdot m + g \cdot n \in N$ for all $f, g \in R$ and $m, n \in M$. Furthermore, $R$-module homomorphism $\phi: M \rightarrow N$ is a map between $R$-modules $M, N$ that is $R$-linear which means that $\phi(f \cdot m + g \cdot n) = f \cdot \phi(m) + g \cdot \phi(n)$ for all $f, g \in R$ and $m,n \in M$.
\subsection*{Isomorphism theorems}
Let us remind isomorphism theorems for $R$-modules.
\theorem\label{iso_thm_module}{\textbf{Isomorphism theorems for $R$-modules}\\
(1) If $L \subset M \subset N$ are submodules then
$$ N/M = (N/L) /(M/L).$$
(2) If $N$ is a module, and $L, M \subset N$ are submodules then
$$ (M+N)/L = M/(M \cap L).$$}
\subsection*{Induced $R[x]$-module by $R$-module homomorphism}
If $\phi$ is an $R$-module endomorphism on $M$, $p(\phi)$ is also an $R$-module endomorphism where $p(x) \in R[x]$. Hence we can consider an $R$-module $M$ as $R[x]$-module with the action $p(x) \cdot m = p(\phi) \cdot m$ where $p(x) \in R[x]$ and $m \in M$.

\newpage
\section{Cayley-Hamilton Theorem}
In linear algebra courses, we proved Cayley-Hamilton theorem for linear operators between vector spaces using concepts of generalized eigen spaces. Cayley-Hamilton theorem can be generalized for some $R$-module homomorphisms. Consider an $R$-module $M$ which is (finitely) generated by $m_1, \cdots, m_n$. An $R$-module homomorphism $\phi: M \rightarrow M$ can be represented by a matrix $A = (a_{ij})$ which has its entries in $R$ because $\phi$ is determined by $\phi(m_j) = \sum_{k=1}^n a_{jk}m_k$.
\definition{\textbf{Determinant of a Matrix}\\
The determinant of an $n \times n$ matrix $A = (a_{jk})$ is 
$$\det A = \sum_{\sigma \in S_n}\mathrm{sign}(\sigma)a_{1\sigma(1)} \cdots a_{n \sigma(n)}.$$}
\definition{\textbf{Characteristic Polynomial}\\
Let $R$ be a commutative ring. Let $A$ be an $n \times n$ matrix with entries in $R$. Then the polynomial $p_A(x) := \det(xI_n-A)$ is called the characteristic polynomial of $A$.}
\proposition{$p_A(x)$ is a monic polynomial of degree $n$ with coefficients in $R$.}
\begin{proof}
Since $xI_n-A$ has its entries in $R[x]$, $p_A(x) = \det(xI_n-A)$ is in $R[x]$. From the definition, each $a_{k\sigma(k)}$ has a degree at most $1$ when $k = \sigma(k)$. (Otherwise, the entry is a constant in $R$.) Hence the identity element of $S_n$ has the unique term with degree $n$ and the term is $(x-a_{11})\cdots (x-a_{nn})$ which is a monic polynomial of degree $n$. Hence $p_A(x)$ is a monic polynomial of degree $n$.
\end{proof}

\theorem\label{cayley_hamilton_thm}{\textbf{Cayley-Hamilton Theorem}\\
Let $M$ be a finitely generated $R$-module. Let $\phi: M \rightarrow M$ be an $R$-module homomorphism. If $p(x)$ is the characteristic polynomial of any matrix representing $\phi$, then $p(\phi) = 0$.}
\begin{proof}
Suppose that $M$ has a generating set $m_1, \cdots, m_n$ and $\phi$ is represented by an $n \times n$ matrix $A = (a_{jk})$ where $\phi(m_j) = \sum_{k=1}^n a_jk m_k$. Hence we have that $\sum_{k=1}^n \delta_{jk}\phi-a_{jk} m_k = 0$ for all $j$ where $\delta_{jk} = 1$ when $j=k$ and $\delta_{jk} = 0$ otherwise. Consider a matrix $\Delta$ having $jk$th entry: $\delta_{jk}\phi - a_{jk}$. Then $\Delta m_j = 0$ for all $j$. Recalling that $\mathrm{adj} \Delta \cdot \Delta = \det \Delta I_n$\footnote{The adjoint matrix of $\Delta$ is $C^T$ where $C$ has $M_{jk}$ $jk$th entry where $M_{jk}$ is the determinant of the $(n-1) \times (n-1)$ matrix which removed $j$th row and $k$th column from $\Delta$.}. Therefore $\det(\Delta) m_j = 0$ for all $j$. Since $\det(\Delta) = p(\phi)$ where $p(x)$ is the characteristic polynomial of $A$, it is proved that $p(\phi) = 0$.
\end{proof}
\corollary{A complex number $\lambda$ is an eigenvalue of $T$ if and only if $\lambda$ is a root of the characteristic polynomial of $T$.}
\begin{proof}
($\Rightarrow$) Let $p(x)$ be the characteristic polynomial of $T$. With Cayley-Hamilton theorem, $p(T) = 0$. Let $v$ be an eigenvector of $T$ having eigenvalue $\lambda$. Then $p(T)v = p(\lambda)v$ hence $p(\lambda)v = 0$. Since $v$ is a nonzero vector, we have $p(\lambda) = 0$ which implies that $\lambda$ is a root of $p(x)$.\\
($\Leftarrow$) Let $\lambda$ be a root of the characteristic polynomial $p(x)$ of $T$. Hence $0 = p(\lambda) = \det(\lambda I_n - A)$ where $A$ is a matrix representing $T$. Hence $\lambda I_n - A$ is not invertible, which implies that there exists a nonzero vector $v \in \mathrm{null}(\lambda I_n - A)$. Then $v$ is an eigenvector having eigenvalue $\lambda$. 
\end{proof}
\exercise{Let $V$ be a finite-dimensional vector space over the field $F$ and $T$ a linear operator on $V$. the minimal polynomial of $T$ is the monic polynomial that generates the annihilator of an $F[x]$-module induced by $T$. Prove that the minimal polynomial of $T$ divides the characteristic polynomial of $T$.}
\begin{solution}
Let $p(x)$ be the characteristic polynomial of $T$ and $q(x)$ be the minimal polynomial of $T$. Since $p(x)$ annihilates $V$, from the definition of the minimal polynomial, $p$ has larger degree than $q$. Hence we can get the remainder $r(x)$ by dividing $p$ by $q$: $p = q \cdot q^\prime+r$. Since $r(T) = p(T)-q(T)\cdot q^\prime(T) = 0$, $r(x)$ must be $0$ because it contradicts to the fact that $q$ is the minimal polynomial otherwise. As a result, $q$ divides $p$.
\end{solution}

\newpage
\section{The Determinant Trick}
\theorem\label{det_trick_thm}{Let $M$ be a finitely generated $R$-module, generated by $n$ elements, and $\phi: M \rightarrow M$ an $R$-module endomorphism. Suppose $I$ is an ideal of $R$ such that $\phi(M) \subset IM$. Then
$$
\phi^n + a_1 \phi^{n-1} + \cdots + a_{n-1}\phi + a_n = 0
$$ for some $a_i \in I$ where $i = 1, \cdots, n$.}
\begin{proof}
With the premise, we have $M = (m_1, \cdots, m_n)$ where $m_1, \cdots, m_n \in M$. For each $j$, we can write $\phi(m_j) = b_j \cdot n_j$ for some $b_j \in I$ and $n_j \in M$ because $\phi(M) \subset IM$. We can write $n_j$ as an $R$-linear combination of $m_1, \cdots, m_n$: $n_j = r_{j1} \cdot m_1 + \cdots r_{jn} \cdot m_n$. Then $A = (a_{jk})$ is a matrix representing $\phi$ where $a_{jk} = b_j r_{jk}$. Every entry of $A$ is in $I$ because $b_j$s are in $I$. The characteristic polynomial $p(x)$ of $A$ has coefficients in $I$. With theorem \ref{cayley_hamilton_thm}, $p(x)$ has degree $n$ and $p(\phi) = 0$.
\end{proof}

\exercise{Show that the coefficients $a_i$ mentioned in the proof are in $I^i$ for $i = 1, \cdots, n$.}
\begin{solution}
Let $A = (a_{jk})$ be the matrix mentioned in the proof of theorem \ref{det_trick_thm}. Considering $x$ and $a_{jk}$s as variables, the each entry of $xI_n - A$ has total degree $1$. Hence the determinant is a homogeneous polynomial with a total degree $n$. Therefore each term $a_k \phi^{n-k}$ has total degree $n$. It implies that $a_k$ has total degree of $k$ considering $a_{jk}$s as variables. Considering $a_{jk}$ as elements in $I$, we can conclude that $a_k \in I^k$.
\end{solution}

\corollary\label{det_trick_cor}{Let $M$ be a finitely generated $R$-module and $I$ an ideal of $R$. If $M = IM$, then there exists an element $s \in R$ such that $s + I = 1+I$ in $R/I$ and $s\cdot M = 0$.}
\begin{proof}
Let us consider the identity map $id_M$ on $M$. By theorem \ref{det_trick_thm}, we have some $a_1, \cdots, a_n \in I$ such that $id_M^n + a_1id_M^{n-1} + \cdots + a_n = 0$. Since the action of $id_M^n +a_1 id^{n-1}_M + \cdots + a_n$ on $M$ (as an $R[x]$-module) is the same as the action of $s = 1 + a_1 + \cdots + a_n$ on $M$ (as an $R$-module), we have $(1+a_1 + \cdots +a_n) \cdot m = 0$ for all $m \in M$. Furthermore, we have $a_1 + \cdots + a_n \in I$ hence $s + I = 1 + I$ in $R/I$.
\end{proof}
\noindent
Theorem \ref{det_trick_thm} and corollary \ref{det_trick_cor} are known as the determinant trick.

\section{Nakayama's Lemma}
\definition{\textbf{Local Ring}\\
A local ring is a commutative ring with a unique maximal ideal.}

\proposition{A ring is local if and only if the set of non-units forms an ideal.}
\begin{proof}
($\Rightarrow$) Take any non-unit $a \in R$. Then $aR$ is an ideal of $R$. Furthermore, $aR$ is a proper ideal of $R$ because $1\in R$ is not in $aR$. By the premise, we have a unique maximal ideal $I$. Since $aR$ is a proper ideal and $I$ is the unique maximal ideal, we have $aR \subset I$. Hence all non-unit elements are in $I$. Also, there are no units in $I$ because it is a proper ideal. Therefore we can conclude that the set of non-units of $R$ is the unique maximal ideal $I$.\\
($\Leftarrow$) Let $I$ be any proper ideal of $R$. There are no units in $I$. Hence $I$ is in the set of non-units. Since the set of non-units is a maximal ideal of $R$ and any proper ideal is in the set, we can conclude that the set of non-units is the unique maximal ideal of $R$. Therefore $R$ is a local ring.
\end{proof}

\theorem{\textbf{Nakayama's Lemma}\\
Let $R$ be a local ring with maximal ideal $I$, and let $M$ be a finitely generated module. If $M = IM$, then $M$ is the zero module.}
\begin{proof}
Since $M = IM$, we can apply the determinant trick. By corollary \ref{det_trick_cor}, we have some $s \in R$ such that $s + I = 1 + I$ and $s$ annihilates $M$. Since $s+I = 1+I$, $s \not\in I$ hence $s$ is a unit in $R$. For any $m$, we have $m = s^{-1}s \cdot m = 0$. Hence $M$ is the zero module.
\end{proof}

\newpage
\corollary\label{c3}{Let $R$ be a local ring with maximal ideal $I$. Let $M$ be an $R$-module and $N$ a submodule of $M$. If $M/N$ is finitely generated and $M = N + IM$, then $M = N$.}
\begin{proof}
The hypothesis $M = N + IM$ implies that $M/N = I(M/N)$. By Nakayama's Lemma, $M/N$ is the zero module, which implies that $M = N$.
\end{proof}

\corollary{Let $R$ be a local ring with maximal ideal $I$ and let $M$ be a finitely generated $R$-module. Then $\{m_1, \cdots, m_k\}$ is a minimal generating set of $M$ if and only if $m_1 + IM, \cdots, m_k + IM$ is a vector space basis of the $R/I$-module $M/IM$. In particular, every minimal generating set of $M$ has the same cardinality.}
\begin{proof}
($\Rightarrow$) Since $m_1, \cdots, m_k$ generates $M$, it is clear that $m_1+IM, \cdots, m_k+IM$ generates $M/IM$ which is a vector space over $R/I$. (Recall that $R/I$ is a field because $R$ is a local ring.) We will show that this set is also linearly independent. Suppose that $\Bar{r_1}(m_1 + IM) + \cdots + \Bar{r_k}(m_k+IM) = 0$ for some $\Bar{r_1}, \cdots, \Bar{r_k} \in R/I$ where $\Bar{r_j} = r_j +I$. Then $r_1 m_1 + \cdots r_km_k$ is in $IM$. Suppose that some $r_j$ is a unit. Then $m_j = r_j^{-1}(r_1 m_1 + \cdots + r_{j-1}m_{j-1} + r_{j+1}m_{j+1} + \cdots + r_km_k)$. This contradicts that $\{m_1, m_2, \cdots, m_k\}$ is a minimal generating set. Hence all $r_j$s are non-units which are in $I$. Hence $\Bar{r_1} = \cdots = \Bar{r_k} = \Bar{0}$. Therefore we can conclude that $m_1+IM, \cdots, m_k+IM$ are linearly independent.\\
($\Leftarrow$) Let $N$ be the submodule of $M$ generated by $m_1, \cdots, m_k$. Then $I(M/N) = (IM+N)/N$. Since $IM+N$ is a submodule of $M$ and the projected image in $M/IM$ is the same as $(IM+N)/IM$. By the premise, $(IM+N)/IM = M/IM$, which implies that $IM + N = M$. Therefore we have $I(M/N) = M/N$. By corollary \ref{c3}, we can conclude that $M = N$ and $M$ is generated by $m_1, m_2, \cdots, m_k$. It is also minimal because of the linear independence of $m_1 + IM, \cdots, m_k+IM$ in the $R/I$-module $M/IM$.\\
Since any basis of a finite dimensional vector space has the same size, we can conclude that every minimal generating set of $M$ has the same cardinality.
\end{proof}

\section*{Reference}
\begin{enumerate}
    \item Undergraduate Commutative Algebra Chapter 2 Modules, Reid
    \item Introduction to Commutative Algebra Chapter 2 Modules, Atiyah and MacDonald
    \item Lecture Note of MATH 850, Prof. Serkan Hosten
\end{enumerate}